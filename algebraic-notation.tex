\documentclass[5pt,a4paper]{article}
\usepackage[left=36mm, top=20mm, right=40mm, bottom=20mm, nohead, nofoot]{geometry}
\usepackage[LSF,T1]{fontenc}
% Packages
\usepackage{chessboard,xskak}   % For chess board
\usepackage{multicol}           % Allows multicols in tables
\usepackage{tikz,everypage}
\usetikzlibrary{backgrounds}

\newcommand*{\AbsolutePosition}[3]{%
    % #1 = x (from south west corner of page
    % #2 = y
    % #3 = content
    \AddThispageHook{%
        \begin{tikzpicture}[remember picture,overlay]
            \draw (current page.south west) ++(#1,#2) node {#3};%
        \end{tikzpicture}
    }%
}
\definecolor{darkgray}{gray}{0.4}
\usepackage{hyperref}
\hypersetup{
    colorlinks=true,
    linkcolor=green,
    filecolor=magenta,      
    urlcolor=darkgray,
}

\usepackage{textpos}
\usepackage{tabularx}           % Intelligent column widths
\usepackage{tabulary}           % Used in header and footer
\usepackage{hhline}             % Border under tables
\usepackage{graphicx}           % For images
\usepackage{xcolor}             % For hex colours
\usepackage[utf8x]{inputenc}    % For unicode character support
\usepackage[T1]{fontenc}        % Without this we get weird character replacements
\usepackage{colortbl}           % For coloured tables
\usepackage{setspace}           % For line height
\usepackage{lastpage}           % Needed for total page number
\usepackage{seqsplit}           % Splits long words.
%\usepackage{opensans}          % Can't make this work so far. Shame. Would be lovely.
\usepackage[normalem]{ulem}     % For underlining links
% Most of the following are not required for the majority
% of cheat sheets but are needed for some symbol support.
\usepackage{amsmath}            % Symbols
\usepackage{MnSymbol}           % Symbols
\usepackage{wasysym}            % Symbols
%\usepackage[english,german,french,spanish,italian]{babel}              % Languages
% Document Info
\author{Dave Child (DaveChild)}
\pdfinfo{
  /Title (chess-algebraic-notation.pdf)
  /Creator (Cheatography)
  /Author (Dave Child (DaveChild))
  /Subject (Chess - Algebraic Notation Cheat Sheet)
}

% Lengths and widths
\addtolength{\textwidth}{6cm}
\addtolength{\textheight}{-1cm}
\addtolength{\hoffset}{-3cm}
\addtolength{\voffset}{-2cm}
\setlength{\tabcolsep}{0.2cm} % Space between columns
\setlength{\headsep}{-12pt} % Reduce space between header and content
\setlength{\headheight}{85pt} % If less, LaTeX automatically increases it
\renewcommand{\footrulewidth}{0pt} % Remove footer line
\renewcommand{\headrulewidth}{0pt} % Remove header line
\renewcommand{\seqinsert}{\ifmmode\allowbreak\else\-\fi} % Hyphens in seqsplit
% This two commands together give roughly
% the right line height in the tables
\renewcommand{\arraystretch}{1.3}
\onehalfspacing

% Commands
\newcommand{\SetRowColor}[1]{\noalign{\gdef\RowColorName{#1}}\rowcolor{\RowColorName}} % Shortcut for row colour
\newcommand{\mymulticolumn}[3]{\multicolumn{#1}{>{\columncolor{\RowColorName}}#2}{#3}} % For coloured multi-cols
\newcolumntype{x}[1]{>{\raggedright}p{#1}} % New column types for ragged-right paragraph columns
\newcommand{\tn}{\tabularnewline} % Required as custom column type in use

% Font and Colours
\definecolor{HeadBackground}{HTML}{333333}
\definecolor{FootBackground}{HTML}{666666}
\definecolor{TextColor}{HTML}{333333}
\definecolor{DarkBackground}{gray}{0.3}
\definecolor{LightBackground}{gray}{0.9}
\renewcommand{\familydefault}{\sfdefault}
\color{TextColor}

% Header and Footer
\thispagestyle{empty}
\pagestyle{fancy}


\begin{document}
\raggedcolumns
\AbsolutePosition{13.5cm}{3.5cm}{\textblock{vk.com/ziferchess}}
\centering
\huge
\textbf{\textcolor{DarkBackground}{\ ALGEBRAIC \ CHESS\ NOTATION\ }}
\par\addvspace{2.5em}

% Set font size to small. Switch to any value
% from this page to resize cheat sheet text:
% www.emerson.emory.edu/services/latex/latex_169.html
\footnotesize % Small font.

\begin{multicols*}{3}

\begin{tabularx}{5.377cm}{X}
\SetRowColor{DarkBackground}
\mymulticolumn{1}{x{5.377cm}}{\bf\textcolor{white}{Algebraic Notation: Squares}}  \tn
% Row 0
\SetRowColor{LightBackground}
\mymulticolumn{1}{x{5.377cm}}{Board is always oriented with a white square at the bottom right.} \tn 
% Row Count 2 (+ 2)
% Row 1
\SetRowColor{white}
\mymulticolumn{1}{x{5.377cm}}{Squares are named from white's perspective} \tn 
% Row Count 3 (+ 1)
% Row 2
\SetRowColor{LightBackground}
\mymulticolumn{1}{x{5.377cm}}{Vertical columns are files, named a-h (always lower case) from left to right} \tn 
% Row Count 5 (+ 2)
% Row 3
\SetRowColor{white}
\mymulticolumn{1}{x{5.377cm}}{Horizontal rows are ranks, named 1-8 from near to far} \tn 
% Row Count 7 (+ 2)
% Row 4
\SetRowColor{LightBackground}
\mymulticolumn{1}{x{5.377cm}}{White king begins on square e1} \tn 
% Row Count 8 (+ 1)
\hhline{>{\arrayrulecolor{DarkBackground}}-}
\end{tabularx}
\par\addvspace{4em}

\begin{tabularx}{5.377cm}{x{1.3731 cm} x{1.69349 cm} x{1.51041 cm} }
\SetRowColor{DarkBackground}
\mymulticolumn{3}{x{5.377cm}}{\bf\textcolor{white}{Algebraic Notation: Pieces}}  \tn
% Row 0
\SetRowColor{LightBackground}
{\bf{Piece}} & {\bf{Code}} & {\bf{Symbol}} \tn 
% Row Count 1 (+ 1)
% Row 1
\SetRowColor{white}
King & K & ♔ \tn 
% Row Count 2 (+ 1)
% Row 2
\SetRowColor{LightBackground}
Queen & Q & ♕ \tn 
% Row Count 3 (+ 1)
% Row 3
\SetRowColor{white}
Rook & R & ♖ \tn 
% Row Count 4 (+ 1)
% Row 4
\SetRowColor{LightBackground}
Knight & N * & ♘ \tn 
% Row Count 5 (+ 1)
% Row 5
\SetRowColor{white}
Bishop & B & ♗ \tn 
% Row Count 6 (+ 1)
% Row 6
\SetRowColor{LightBackground}
Pawn & {[}no letter{]} & ♙ \tn 
% Row Count 7 (+ 1)
\hhline{>{\arrayrulecolor{DarkBackground}}---}
\SetRowColor{LightBackground}
\mymulticolumn{3}{x{5.377cm}}{Pieces are always uppercase. \newline * In chess problems, "S" is used to represent the Knight.}  \tn 
\hhline{>{\arrayrulecolor{DarkBackground}}---}
\end{tabularx}
\par\addvspace{4em}

\begin{tabularx}{5.377cm}{p{0.4577 cm} x{1.00694 cm} x{3.11236 cm} }
\SetRowColor{DarkBackground}
\mymulticolumn{3}{x{5.377cm}}{\bf\textcolor{white}{Algebraic Notation: Moves}}  \tn
% Row 0
\SetRowColor{LightBackground}
\mymulticolumn{3}{x{5.377cm}}{The notation for a move indicates which piece is moved, and to where:} \tn 
% Row Count 2 (+ 2)
% Row 1
\SetRowColor{white}
 & {\bf{Qa3}} & Queen moves to a3 \tn 
% Row Count 3 (+ 1)
% Row 2
\SetRowColor{LightBackground}
 & {\bf{Kh6}} & King moves to h6 \tn 
% Row Count 4 (+ 1)
% Row 3
\SetRowColor{white}
 & {\bf{b4}} & Pawn moves to b4 \tn 
% Row Count 5 (+ 1)
% Row 4
\SetRowColor{LightBackground}
\mymulticolumn{3}{x{5.377cm}}{Where two identical pieces could move to the same square, the piece name is followed by its original file (or rank where the file is the same), like so:} \tn 
% Row Count 9 (+ 4)
% Row 5
\SetRowColor{white}
 & {\bf{Rba3}} & Rook on b file moves to a3 \tn 
% Row Count 10 (+ 1)
% Row 6
\SetRowColor{LightBackground}
 & {\bf{N4f2}} & Knight on rank 4 moves to f2 \tn 
% Row Count 12 (+ 2)
% Row 7
\SetRowColor{white}
 & {\bf{cxd5}} & Pawn on c file takes d5 \tn 
% Row Count 13 (+ 1)
\hhline{>{\arrayrulecolor{DarkBackground}}---}
\end{tabularx}
\par\addvspace{4em}

\begin{tabularx}{5.377cm}{p{0.94563 cm} x{4.03137 cm} }
\SetRowColor{DarkBackground}
\mymulticolumn{2}{x{5.377cm}}{\bf\textcolor{white}{Algebraic Notation: Symbols}}  \tn
% Row 0
\SetRowColor{LightBackground}
x & Piece taken \tn 
% Row Count 1 (+ 1)
% Row 1
\SetRowColor{white}
e.p. & Piece taken en passant \tn 
% Row Count 2 (+ 1)
% Row 2
\SetRowColor{LightBackground}
+ & Check \tn 
% Row Count 3 (+ 1)
% Row 3
\SetRowColor{white}
\# & Checkmate \tn 
% Row Count 4 (+ 1)
% Row 4
\SetRowColor{LightBackground}
= & Pawn promotion * \tn 
% Row Count 5 (+ 1)
% Row 5
\SetRowColor{white}
0-0 & Castle King-side \tn 
% Row Count 6 (+ 1)
% Row 6
\SetRowColor{LightBackground}
0-0-0 & Castle Queen-side \tn 
% Row Count 7 (+ 1)
% Row 7
\SetRowColor{white}
1-0 & White win \tn 
% Row Count 8 (+ 1)
% Row 8
\SetRowColor{LightBackground}
$\frac{1}{2}$-$\frac{1}{2}$ & Draw \tn 
% Row Count 9 (+ 1)
% Row 9
\SetRowColor{white}
0-1 & Black win \tn 
% Row Count 10 (+ 1)
% Row 10
\SetRowColor{LightBackground}
(=) & Draw offered \tn 
% Row Count 11 (+ 1)
\hhline{>{\arrayrulecolor{DarkBackground}}--}
\SetRowColor{LightBackground}
\mymulticolumn{2}{x{5.377cm}}{* e8=Q means e-file pawn promoted to Queen. The equals is often omitted.}  \tn 
\hhline{>{\arrayrulecolor{DarkBackground}}--}
\end{tabularx}
\par\addvspace{4em}

\begin{tabularx}{5.377cm}{p{0.4977 cm} x{4.4793 cm} }
\SetRowColor{DarkBackground}
\mymulticolumn{2}{x{5.377cm}}{\bf\textcolor{white}{Algebraic Notation: Example}}  \tn
% Row 0
\SetRowColor{LightBackground}
\mymulticolumn{2}{x{5.377cm}}{1. e4 c5} \tn 
% Row Count 1 (+ 1)
% Row 1
\SetRowColor{white}
 & White pawn to e4; Black pawn to c5 \tn 
% Row Count 2 (+ 1)
% Row 2
\SetRowColor{LightBackground}
\mymulticolumn{2}{x{5.377cm}}{2. Nf3 d6} \tn 
% Row Count 3 (+ 1)
% Row 3
\SetRowColor{white}
 & White knight to f3 \tn 
% Row Count 4 (+ 1)
% Row 4
\SetRowColor{LightBackground}
\mymulticolumn{2}{x{5.377cm}}{3. Bb5+ Bd7} \tn 
% Row Count 5 (+ 1)
% Row 5
\SetRowColor{white}
 & White bishop to b5, Black in check \tn 
% Row Count 6 (+ 1)
% Row 6
\SetRowColor{LightBackground}
\mymulticolumn{2}{x{5.377cm}}{4. Bxd7+ Qxd7} \tn 
% Row Count 7 (+ 1)
% Row 7
\SetRowColor{white}
 & White bishop takes black bishop on d7, black in check; Queens takes d7 bishop \tn 
% Row Count 10 (+ 3)
% Row 8
\SetRowColor{LightBackground}
\mymulticolumn{2}{x{5.377cm}}{5. c4 Nc6} \tn 
% Row Count 11 (+ 1)
% Row 9
\SetRowColor{white}
\mymulticolumn{2}{x{5.377cm}}{6. Nc3 Nf6} \tn 
% Row Count 12 (+ 1)
% Row 10
\SetRowColor{LightBackground}
\mymulticolumn{2}{x{5.377cm}}{7. 0-0 g6} \tn 
% Row Count 13 (+ 1)
% Row 11
\SetRowColor{white}
 & White castles king-side \tn 
% Row Count 14 (+ 1)
% Row 12
\SetRowColor{LightBackground}
\mymulticolumn{2}{x{5.377cm}}{8. d4 cxd4} \tn 
% Row Count 15 (+ 1)
% Row 13
\SetRowColor{white}
 & White d4 pawn taken by c-file black pawn \tn 
% Row Count 17 (+ 2)
% Row 14
\SetRowColor{LightBackground}
\mymulticolumn{2}{x{5.377cm}}{9. Nxd4 Bg7} \tn 
% Row Count 18 (+ 1)
% Row 15
\SetRowColor{white}
\mymulticolumn{2}{x{5.377cm}}{10. Nde2 Qe6} \tn 
% Row Count 19 (+ 1)
% Row 16
\SetRowColor{LightBackground}
 & White knight on d-file to e2 \tn 
% Row Count 20 (+ 1)
\hhline{>{\arrayrulecolor{DarkBackground}}--}
\SetRowColor{LightBackground}
\mymulticolumn{2}{x{5.377cm}}{Moves are first ten from Kasparov vs the World, \href{http://bit.ly/1fOcfIY}{http://bit.ly/1fOcfIY}}  \tn 
\hhline{>{\arrayrulecolor{DarkBackground}}--}
\end{tabularx}
\par\addvspace{4em}

\begin{tabularx}{5.377cm}{p{0.4977 cm} x{4.4793 cm} }
\SetRowColor{DarkBackground}
\mymulticolumn{2}{x{5.377cm}}{\bf\textcolor{white}{Algebraic Notation: Annotations}}  \tn
% Row 0
\SetRowColor{LightBackground}
!! & Extremely strong move, often game-winning \tn 
% Row Count 2 (+ 2)
% Row 1
\SetRowColor{white}
! & Great move \tn 
% Row Count 3 (+ 1)
% Row 2
\SetRowColor{LightBackground}
!? & Speculative move, possibly strong but more analysis needed \tn 
% Row Count 5 (+ 2)
% Row 3
\SetRowColor{white}
?! & Dubious move, possibly weak but more analysis needed \tn 
% Row Count 7 (+ 2)
% Row 4
\SetRowColor{LightBackground}
? & Bad move \tn 
% Row Count 8 (+ 1)
% Row 5
\SetRowColor{white}
?? & Blunder, equivalent to hanging a piece \tn 
% Row Count 10 (+ 2)
% Row 6
\SetRowColor{LightBackground}
+- & White is winning \tn 
% Row Count 11 (+ 1)
% Row 7
\SetRowColor{white}
+/- & White has a significant edge \tn 
% Row Count 12 (+ 1)
% Row 8
\SetRowColor{LightBackground}
+/= & White has a small edge \tn 
% Row Count 13 (+ 1)
% Row 9
\SetRowColor{white}
= & Equality \tn 
% Row Count 14 (+ 1)
% Row 10
\SetRowColor{LightBackground}
∞ & Unclear advantage \tn 
% Row Count 15 (+ 1)
% Row 11
\SetRowColor{white}
=/+ & Black has a small edge \tn 
% Row Count 16 (+ 1)
% Row 12
\SetRowColor{LightBackground}
-/+ & Black has a significant edge \tn 
% Row Count 17 (+ 1)
% Row 13
\SetRowColor{white}
-+ & Black is winning \tn 
% Row Count 18 (+ 1)
% Row 14
\SetRowColor{LightBackground}
□ & Only move available \tn 
% Row Count 19 (+ 1)
\hhline{>{\arrayrulecolor{DarkBackground}}--}
\SetRowColor{LightBackground}
\mymulticolumn{2}{x{5.377cm}}{From the excellent guide at \href{http://bit.ly/1iSkXch}{http://bit.ly/1iSkXch}}  \tn 
\hhline{>{\arrayrulecolor{DarkBackground}}--}
\end{tabularx}
\par\addvspace{4em}


\setlength{\parindent}{-4em}
\begin{tabularx}{5.377cm}
\raggedleft
\setchessboard{
smallboard,
color=black,
clearboard}
\scalebox{1.22}{
\chessboard[
showmover=false,
pgfstyle=color,
pgfstyle=
{[base,at={\pgfpoint{0pt}{-0.4ex}}]text},
text= \fontsize{1.2ex}{1.2ex}\bfseries
\sffamily\currentwq,
markboard,
border=false,linewidth=0.4pt,padding=0.4pt,pgfborder,linewidth=1pt,padding=2pt,pgfborder,boardfontfamily=times,boardfontseries=b,
color=LightBackground,
pgfstyle=color,
padding=-0.01em,
trimtocolor=black,
backboard,]}
\end{tabularx}


% That's all folks

\end{multicols*}

\end{document}
